\documentclass{article}
\usepackage{apacite,url}

\title{EEG toolbox documentation}
\author{Computational Memory Lab}

%%% Plan for tutorial:
% 1) intro to data setup
% 2) extraction, alignment and rereferencing
% 3) simple ERPs
% 4) oscillatory analysis
% 5) aggregation stats for oscillatory analysis (bootstrap function)
% 6) plotting: topoplots with EEGLAB and tal3d for iEEG

\begin{document}
\maketitle

% Note: need to provide sample data with this documentation which can
% then serve as a tutorial

\textbf{Note:} for all of the functions in the EEG toolbox, you can
get help by typing 'help $<$functionname$>$' at the matlab
prompt. This will show you what the function does and what the input
and output should be.

\section{Getting started}
The EEG toolbox operates on data in a particular directory structure,
where the data is saved in separate files for each channel, in a
separate directory for each subject. There is a top-level directory
often referred to as ``basedir'', which could be named after your
experiment, for example ``ddm\_eeg''. In that basedir, there is a
directory for every subject, e.g., ``DOT002''. Every subject has
various standard directories, most importantly a directory with events
structures (will be explained later on). This directory can be named
``events''. It is important to place all your data under an ``eeg''
directory, which can hold files with information relevant to all data
analyses such as specifications for wavelet and multitaper parameters.

\begin{verbatim}
~/eeg/
  eeganalparams.txt
  ddm_eeg/
    DOT001/
      events/
      behavioral/
      eeg.reref/
      eeg.noreref/
        params.txt
      raw/
    DOT002/
    DOT003/
\end{verbatim}

If you are using Cogniscan data, the EEG toolbox contains a function to
split the .bin files into single channel files (\verb7cogniscan_split.m7). A
more generic function that reads binary files is \verb7spliteeg.m7,
otherwise, you can write your own function along these lines. This
will, importantly, create a file \verb7params.txt7, with parameters
like the sample rate. Another important file is
\verb7eeganalparams.txt7 in the \verb3~\eeg\3 directory, which
contains analysis methods parameters, e.g., wavelet width. Your
\verb5eeganalparams.txt5 file can look for example like this:\\
\begin{verbatim}
freqs (2^(1/8)).^(8:56)
width 6
bandwidth 1
windowsize 0.3
\end{verbatim}
In the above file, \verb4freqs4 is the vector of frequencies that will
be analyzed in the oscillatory analysis scripts, and \verb5width5 is
the width of the wavelets (for which we generally use
6~\cite{TallEtal97,CaplEtal01}).

In general if you want to look at a file with EEG data, you can use
the function \verb5look.m5, which takes as input argument an EEG
filename and a channel, and gets basically the contents of the file as
a vector and plots it on the screen.
 
\section{Events}
The EEG toolbox is built around events structures, which are matlab
structures that contain your behavioral data, and that can be used to
retrieve the associated EEG data. You will have to write your own
matlab script to extract events from your behavioral logs, which can
be modeled on the sample script \verb7createDDMevents.m7. The crucial
fields that every events structure should have in order to function in
the EEG toolbox are a field containing the events time in the units of
your behavioral data file (e.g., milliseconds). You can then select
subsets of events using the \verb7filterStruct.m7 function, which
takes in a logical expression, e.g., to get all CUE events we use
\begin{verbatim}
cueEvents = filterStruct(events,'strcmp(type,''CUE'')')
\end{verbatim}
 where 'events' is the name of the events structure we have loaded
 with
\begin{verbatim}
events=loadEvents('~/mms/data/DOT001/events/events.mat')
\end{verbatim}
 (an alternative way of loading events is
 \verb5events=loadSubjEvents5, which takes as input arguments a
 basedir (see above), a cell array with subjects (e.g.,
 subj={'DOT001','DOT002','DOT004'}) and the name of events files
 (e.g., 'beh/events.mat')). These cueEvents can then be used in
 subsequent EEG toolbox functions like \verb5erp.m5 (see below).

Another very useful function is \verb5getStructField5, which gives you
the contents of a particular field in your events structure in an
array, e.g., 
\begin{verbatim}
isCorrect=getStructField(events,'correct')
\end{verbatim}
gives you a binary vector that for every event tells you whether a
subject got it right. If the field contains strings, the output of
this function will be a cell array. Similarly, the \verb5inStruct.m5
function will tell you for a certain logical expression, e.g.,
'iscorrect==0' whether that is true for every event (so it yields a
binary vector with the length of the events structure). Sometimes people get errors when using filterStruct, which could be caused by empty fields. If the length of the result of inStruct.m is less than the length of your events struct, then that might be the issue.


\section{Align}
Once you have your events structures and EEG split out by channels,
you are ready for aligning these events with the EEG data. This will
create the fields ``eegfile'' and ``eegoffset'' in your events, which
are crucial for the rest of the EEG toolbox to work. The exact
functions to use for aligning are exemplified in the \verb3postprocess_cogniscan.m3
function, and include \verb5runalign.m5 and \verb6logalign.m6.\footnote{To use this script successfully you will need to install EEGLAB and Fieldtrip as well. Fieldtrip can be found at \url{http://neuroimaging.ruhosting.nl/fieldtrip/doku.php}} This
will basically take the extracted pulses from the EEG data and the
pulse sending logs from your behavioral data and use regression to
line those up. When you are dealing with iEEG data, you will have to
extract the pulses from an analog channel, and this is best done using
\verb5alignTool.m5, which has a gui to extract pulses and also to put
all the correct files together for the alignment.  Although cogniscan produces also an analog channel with sync pulses, this channel can easily be thresholded to get the sync pulses, which is the method used by \verb5postprocess_cogniscan.m5.

After you have
aligned the data, you probably want to run an eyeblink detection
script, which we have called \verb5addArtifacts.m5. This basically
looks for large deflections in the fast and slow moving averages of
the EEG data. In the cogniscan data, you probably want to call it as
follows: 
\begin{verbatim}
addArtifacts('events_1.mat',{1},100,0);
\end{verbatim}
which basically uses the channels above and below the eye in bipolar
to find these deflections.

\section{ERP}
Plotting ERPs using the EEG toolbox is really simple: use the
\verb4erp.m4 function. First select an interesting set of events as
described above, e.g.,
\begin{verbatim}
corrEvs=filterStruct(events,'correct')
\end{verbatim}
 Then plug those into
the ERP generating function, e.g.,
\begin{verbatim}
erp_chan10=erp(10,corrEvs,600,-100,500,[-100 0],[58 62],'stop',1,5)
\end{verbatim}
A couple of notes are in order: the durations are specified in
samples, so in the case of a samplerate of 500 Hz we often want to
look at 1 second, with a 200-ms baseline (100 samples). The next
variable is the buffer, for which we usually want a second, which is
necessary to avoid filtering or convolution artifacts. The variable
\verb6[-100 0]6 indicates that we use the 100 ms before the event as a
baseline. Then really important is filtering out the line noise, which
is why we use \verb3[58 62]3, of a 'stop' (bandpass) filter, with one
filtering iteration. The last parameter is the kurtosis threshold,
which we use to throw out artifacts~\cite{DeloMake04}. 5 is usually a
pretty good threshold for scalp and iEEG in our experience. You can
use the function \verb6ga_erp.m6 to compute a grand average ERP over
subjects, which also gives you errorbars, and can compute the ERP both
treating subjects independently as well as treating all data as one
subject (see the help for more info). 

In order to determine the significance of differences in ERP
amplitude, you can use the \verb5bootstrap.m5 function. This function
takes in as input arguments the number of bootstrap iterations
(usually we use 1000), then either one or two vectors/matrices of
data. The data can for example be the difference in ERP amplitude
between condition 1 and condition 2 for every subject or every event
at every point in time. The bootstrap function will create a
distribution of results of your statistic (a one-sample t-test or
signed rank test for one-sample data, or a two-sample t-test or
ranksum test for two-sample data) for your data (in the one-sample
case, you sign of the data is randomly swapped, in the two-sample
case, the conditions of your data are swapped). You can then examine
how far your observed test statistic falls outside the randomized
distribution (the observed test statistic is the second output
argument of the \verb5bootstrap.m5 function).

\section{Power (wavelet and multitaper)}
Similar to ERPs, our wavelet processing takes in a set of events and a
channel in order to provide you with an events $\times$ frequencies $\times$ samples
(time) array with power amplitude values (and if desired also
phases). The most useful function here is probably
\verb5getphasepow.m5, which gives you basically that data. A lower
level function, if you do not have events, but instead plain EEG, is
\verb5multienergyvec5. An example of calling \verb6getphasepow.m6 is:
\begin{verbatim}
[phase,pow] = getphasepow(10,corrEvs,1000,0,1000,'filtfreq',[58
62],'filttype','stop','kthresh',5)
\end{verbatim}
which uses very similar parameters to the erp function, but now takes
in durations in milliseconds. Here you do not do the baseline
correction, so there is no variable for that (and probably most of the
time you want your offset to be 0). Plotting a simple power
spectrogram can be done by taking the mean over events (where you
probably want to take a log10 of the power in order to be able to see
anything, to counteract the 1/f falloff of the power spectrum). An
example would be 
\begin{verbatim}
imagesc(squeeze(mean(log10(pow))))
\end{verbatim}
In order to make power comparable across frequencies and subjects, it
is often useful to look at z-transformed power. The function that does
this is \verb5ztrans_pow.m5. The input arguments are very similar to
\verb6getphasepow.m6, except that they are inserted in a
structure. Different is optional info you can give the function about
pieces of data you'd like to give it to compute baseline power (e.g.,
times when the subject is staring at a fixation cross). If you do not
give separate baseline info, then the baseline is the mean and
standard deviations over all events you give it as input.

For multitapers, we use very similar functions, where the main
function is \verb5mtphasepow.m5, which takes as input again the
channel, events and desired durations. The optional parameters are
slightly different: the bandwidth and windowsize of the
multitapers. We have found that a bandwidth of 1 and windowsize of .3
work quite well, even though that uses only one taper. Note that you
always have to adapt both windowsize and bandwidth together (see
~\citeA{RaghEtal01,vanVEtal07a} for more info). 

In order to asses significance of power results, we use mainly two
randomization methods (you can also use the False Discovery rate
procedure~\cite{BenjHoch95}, with the function \verb5fdr.m5). The
first randomization method is called the ``sum-z
method''~\cite{GibbShan87,SedeEtal07b}, where we compute a
bootstrapped test statistic for every electrode, frequency and
timepoint/timebin within every subject. We then combine across
subjects by adding up the \verb5norminv.m5 of every p-value for the
actual statistics as well as the bootstrapped distributions, and check
where the actual statistic falls in that distribution, which is the
final p-value across subjects for that frequency, timebin and
electrode.


\section{Pepisode}
Pepisode is a power analysis developed by Jeremy Caplan in our
lab~\cite{CaplEtal01}, and basically it computes the fraction of time
an oscillation exceeding the background activity is present in your
events. Calling these functions is very similar to calling power
functions. However, it requires a pre-processing step, in which for
every piece of EEG data, it is determined whether there are
significant oscillations present at every frequency (significant
oscillations exceed a duration and an amplitude threshold). This
pre-processing function is called \verb4calcPepisode.m4, and takes as
input arguments a set of events, a channel, and an output directory
name in which the pre-processed data will be placed (matrices with
zeros or ones for every point in time and every frequency, indicating
whether there is a significant oscillation). The function
\verb5getuvec_ms5 will then return this binary vector for your events
of interest and for the specified duration. The variable bgThreshold
indicates at what fraction of the mean oscillation amplitude over the
whole EEG file the threshold should be set (usually set to 95\%). You
can take the mean of this binary vector over time in order to
determine what fraction of your time interval of interest is taken up
by this oscillation.

\section{Plotting}
A useful function for plotting is \verb6publishfig.m6, which makes the
figure such that it has larger axis labels etc, which is useful for
publishing purposes. Note that it removes the title, so apply the
title after running this function.

For plotting the topography of effects for scalp EEG, we use
EEGLAB~\cite{DeloMake04}, \url{http://sccn.ucsd.edu/eeglab/}. The function that we use
that is \verb4headplot.m4, which requires as input a specification of where
the electrodes are located (for our cogniscan system that is a file named
'Edited\_Standard-10-5-Cap385.sfp') and an array with a value to be plotted for every
electrode. There are also other plotting scripts in the \verb4tal4 directory of
the EEG toolbox, which are older and not as accurate in general.

\section{Simulation}
If you're interested in testing new EEG analysis tools, you can
generate fake EEG of a specified number of events, duration,
samplerate and amplitude using \verb5getfakeEEG.m5, to which you can
add sinusoidal signals of specified amplitude, phase, duration and
time using \verb5addSignal.m5 or \verb5addSignalProp.m5 (where the
second version adds a signal with an amplitude that is a proportion of
the background EEG and the first one specifies an absolute
amplitude). These scripts were used in~\cite{vanVEtal07a}.

\section{Running scripts on the cluster}
To run Matlab scripts on the cluster, there are a few useful eeg
toolbox functions. Basically you need to write a script that loops
over some process that is independent of others, such as doing a
computation for every electrode (looping over them) and storing the
results of each in a separate file. You would probably use code
similar to this:\\
\begin{verbatim}
   for c = 1:128
    chanFile = sprintf('~/mms/data/datFiles/myAnalysis%.03d.mat',c);
    if ~exist(chanFile)
       lockFile(chanFile);
       << do computation>>
       << save data in chanFile >>
       releaseFile(chanFile);
    end
   end
\end{verbatim}


You can find the data for the Dots experiment here:\\
\verb7/mounts/jukebox_static/people/mkvan/eeg/ddm_eeg7

\bibliographystyle{apacite}
\bibliography{memlab}


\end{document}
